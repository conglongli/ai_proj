\section{Introduction}

Traditionally, circuit switching and packet switching have been
considered as alternative design choices, each with their own advantages
and disadvantages. Packet switches are efficient at multiplexing traffic
across a large number of ports, while circuit switches can often service
higher line rates in a more cost-effective manner, especially when
combined with optical links. However, as optical circuit switching
technology advances, the reconfiguration time (i.e., the amount of time
it takes to alter the input-to-output mapping) is swiftly decreasing,
thus increasingly blurring the division between the packet and circuit
regimes.

Indeed, a range of new datacenter switch designs take advantage of this
trend, and propose to schedule appropriately large traffic demands via a
high-bandwidth circuit switch and handle any remaining traffic with a
slower packet switch. However, all recent proposals for such hybrid
designs presume the existence of an omniscient scheduling oracle that
can compute switch configurations and map traffic to them in an optimal
fashion. Recently, Liu \textit{et al.\ }~\cite{Liu:2015} proposed a
hybrid switch scheduling algorithm, Solstice, that is both
highly effective at scheduling datacenter-like traffic workloads and has
practical computational overheads.

In this project, we improved the Solstice scheduling algorithm by
introducing an indirection technique that reduces the number of
configurations needed for scheduling. This proposed technique
aims to redirect traffic through underutilized links. Using this technique,
traffic demand can be served by pairs of existing links, utilizing
multi-hop routes to increase usage of underutilized links.
This technique is evaluated in simulation as well as compared with ILP
formulations for switch configurations.

The rest of this proposal is organized as follows.
Section~\ref{sec:related} discusses Solstice and other related work.
Section~\ref{sec:indirection} describes the proposed indirection
technique. Finally, Section~\ref{sec:goal} describes the proposed evaluation plan
and goals.

\section{Related Work}
\label{sec:related}
Recently, researchers have proposed hybrid datacenter network architectures
that offer higher throughputs at lower cost by combining switching
technologies. In particular, recent proposals suggest employing
highspeed optical~\cite{Chen:2012, Farrington:2010, Wang:2010} or
wireless~\cite{Halperin:2011, Kandula:2009, Zhou:2012} networks
configured to service the heavy flows, while passing the remainder of
the traffic through a traditional, relatively underprovisioned
packet-switched network. 

As technology trends usher in dramatically faster reconfiguration times,
the distinction between packet and circuit is blurred, and ever smaller
flows can take advantage of a hybrid fabric. This trend will soon allow
servicing the bulk of the traffic through a rapidly reconfigurable
optical switch~\cite{Porter:2013}, leaving a relatively minor portion to
be serviced by the packet network~\cite{Liu:2014}. While the potential
cost savings that hybrid technologies could realize is large, the design
space for scheduling resources in the hybrid regime is not yet well
understood. What range of traffic demands can be scheduled for a given
switch design and how can this schedule be computed efficiently? These
questions are not addressed by the existing switch scheduling literature.

Many of the classical approaches to scheduling for switches with
non-trivial reconfiguration delays divide the offered demand into two
parts: an initial, heavy-weight component that is served by O(N) highly
utilized configurations with significant durations, and a second,
residual component that is serviced by a similar number of short,
under-utilized schedules. The recent Solstice scheduling algorithm~\cite{Liu:2015}
exploits the skewed nature of datacenter traffic
patterns to create a small number of configurations with long durations
that minimize the penalty for reconfiguration and leave only a small
amount of residual demand to be serviced by a low-speed (and lowcost)
unconstrained packet switch.

\section{Scheduling Without Indirection}
\label{sec:noindirection}
Solstice introduces a method of calculating a schedule for the circuit switch
and determine what data to send to the packet switch in order to satisfy all
demand while maximizing utilization.

\subsection{Notation}
The switch in use is a \textbf{hybrid switch} composed of
a circuit switch and a packet switch. The circuit switch has
a link rate of $r_c$ bits/second with a reconfiguration time of $\delta$.
The packet switch has a link rate of $r_p \ll r_c$, bits/second.

\textbf{Demand} is the amount of data that must be sent between any two ports.
It is expressed as a matrix $D$ of size $n \times n$,
where rows are sources and columns are destinations.
The amount of demand from port $a$ to port $b$ is $D_{a, b} \in \mathbf{R}^{+}_0$

A \textbf{Schedule} is a set of configurations $\{P_1, P_2, \dots, P_m\}$, durations
$\{t_1, t_2, \dots, t_m\}$, and packet switch utilization matrices ${E_1, E_2, \dots, E_m}$.
$P_{i_{a, b}}$ is an $n \times n$ binary matrix, where $P_{i_{a, b}} = 1$ indicates that
port $a$ can second to port $b$ when the switch is using the configuration $P_i$.
Each row and each column in $P_i$ has exactly one 1.
For each configuration there is an associated set of data $E_i$ to be sent via the
packet switch during that configuration.
$E_{i_{a, b}} = k$ means $k$ bits are sent from $a$ to $b$ through the packet switch instead
of the circuit switch during configuration $i$.


\subsection{Considerations}
We seek to minimize transmission time, which is equivalent to maximizing utilization.
Additionally, we wish to fully schedule existing demand before scheduling
new demand to prevent starvation.

We define {\it total time} to be the amount of time scheduled
on the circuit switch plus the amount of time spent switching.
\[T = \left(\sum_{i=1}^{m} t_i\right) + m\delta.\]

In the hybrid switch, the packet switch can only be utilized during times when
the packet switch is utilized. Thus, all outbound links $i$ can admit the following
amount of data:
\[\sum_{j=1}^{m} E_{i,j} \quad\leq\quad r_p T.\]
Inbound links are similarly constrained.

Demand is satisfied when the total data served by the hybrid switch is at least
as large as the demand:
\[E + \sum_{i=1}^{m} r_c t_i P_i \quad\geq\quad D.\]

We formulate this problem as an ILP, shown in figure~\ref{fig:hybrid-optimization}.

\begin{figure}[t]
\small
\centering
\begin{mdframed}
\[
\min \left(\sum_{i=1}^{n!} t_i\right) + m\delta
\]
\begin{tabbing}
\textbf{subject to:}\\[4pt]
1) $E + \sum_{i=1}^{n!} r_c t_i P_i \geq D$\\[4pt]
2) $\forall i: \sum_{j=1}^{n} E_{i,j} \leq r_p T$\\[4pt]
3) $\forall j: \sum_{i=1}^{n} E_{i,j} \leq r_p T$\\[4pt]
4) $\forall i: t_i \leq \textit{max}(D) l_i$
%5) $\forall i: t_i \geq l_i$
\end{tabbing}
\end{mdframed}
\caption{An ILP to find optimal schedules for a hybrid switch.}
\label{fig:hybrid-optimization}
\end{figure}

\section{The Indirection Technique}
\label{sec:indirection}
We improved the Solstice scheduling
algorithm by introducing a indirection technique that reduces the number
of configurations needed for the scheduling. Originally, Solstice
schedules each traffic demand by a direct path, which means that a new
configuration is required to fulfill a demand that has different sources
or destinations. Our technique schedules part of demand traffic
indirectly. For example, port a has 5 demand to port b and 20
demand to port c; port c has 20 demand to port b. Solstice will require
3 configurations for these demands. However, we could let port a send
all 25 demand to port c and let port c send 25 demand to port b. In this
way we reduce one configuration time, which is nontrivial for optical
switches. 

\section{Evaluation}
\label{sec:goal}
First, evaluated our indirection technique by integer
linear programming (ILP) formulations. In the paper, Liu \textit{et al.\ }~\cite{Liu:2015}
provided a ILP formulation of the scheduling problem without indirection
considerations. We built a new formulation adding indirection to Liu's formulation.
Then used Gurobi~\cite{gurobi}
ILP solver to solve scheduling problems with the two formulations in order
to find the optimal improvement by indirection.

Then we evaluated our indirection techinque in simulation. We received the
simulator of the Solstice algorithm from the authors, and introduced
our techinque into the simulator to compare it with the original algorithm.

%The simulation results will be comparable in terms of number of configurations,
%and the total time to serve all demands.

