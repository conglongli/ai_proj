\section{Introduction}

Traditionally, circuit switching and packet switching have been
considered as alternative design choices, each with their own advantages
and disadvantages. Packet switches are efficient at multiplexing traffic
across a large number of ports, while circuit switches can often service
higher line rates in a more cost-effective manner, especially when
combined with optical links. However, as optical circuit switching
technology advances, the reconfiguration time (i.e., the amount of time
it takes to alter the input-to-output mapping) is swiftly decreasing,
thus increasingly blurring the division between the packet and circuit
regimes.

Indeed, a range of new datacenter switch designs take advantage of this
trend, and propose to schedule appropriately large traffic demands via a
high-bandwidth circuit switch and handle any remaining traffic with a
slower packet switch. However, all recent proposals for such hybrid
designs presume the existence of an omniscient scheduling oracle that
can compute switch configurations and map traffic to them in an optimal
fashion. Recently, Liu \textit{et al.\ }~\cite{Liu:2015} recently
proposed a hybrid switch scheduling algorithm, Solstice, that is both
highly effective at scheduling datacenter-like traffic workloads and has
practical computational overheads when doing so.

In this proposoal, we propose to improve the Solstice scheduling
algorithm by introducing an indirection technique that reduce the number
of configuration needed for the scheduling. This proposed technique aims
to redirect part of demand traffics to the existing underutilized
scheduled links. In this way, those emands are indirectly served by
pairs of existing links and won't require additional configuration time.
We plan to evaluate this technique by both simulations and comparisons
with integer linear programming (ILP) formulations.

The rest of this proposal is organized as follows.
Section~\ref{sec:related} discusses Solstice and other related work.
Section~\ref{sec:indirection} describes the proposed indirection
technique. Finally, Section~\ref{sec:goal} describes the proposed evaluation plan
and goals.

\section{Related Work}
\label{sec:related}
Recently, researchers proposes hybrid datacenter network architectures
that offer higher throughputs at lower cost by combining switching
technologies. In particular, recent proposals suggest employing
highspeed optical~\cite{Chen:2012, Farrington:2010, Wang:2010} or
wireless~\cite{Halperin:2011, Kandula:2009, Zhou:2012} networks
configured to service the heavy flows, while passing the remainder of
the traffic through a traditional, relatively underprovisioned
packet-switched network. 

As technology trends usher in dramatically faster reconfiguration times,
the distinction between packet and circuit is blurred, and ever smaller
flows can take advantage of a hybrid fabric. This trend will soon allow
servicing the bulk of the traffic through a rapidly reconfigurable
optical switch~\cite{Porter:2013}, leaving a relatively minor portion to
be serviced by the packet network~\cite{Liu:2014}. While the potential
cost savings that hybrid technologies could realize is large, the design
space for scheduling resources in the hybrid regime is not yet well
understood. What range of traffic demands can be scheduled for a given
switch design and how can this schedule be computed efficiently? These
questions are not addressed by the existing switch scheduling literature.

Many of the classical approaches to scheduling for switches with
non-trivial reconfiguration delays divide the offered demand into two
parts: an initial, heavy-weight component that is served by O(N) highly
utilized configurations with significant durations, and a second,
residual component that is serviced by a similar number of short,
under-utilized schedules. The recent Solstice scheduling algorithm~\cite{Liu:2015}
exploits the skewed nature of datacenter traffic
patterns to create a small number of configurations with long durations
that minimize the penalty for reconfiguration and leaves only a small
amount of residual demand to be serviced by a low-speed (and lowcost)
unconstrained packet switch.

\section{The Indirection Technique}
\label{sec:indirection}
In this proposoal, we propose to improve the Solstice scheduling
algorithm by introducing a indirection technique that reduce the number
of configuration needed for the scheduling. Originally, Solstice will
schedule each traffic demand by a direct path, which means that a new
configuration is required to fulfill a demand that has different sources
or destinations. Our technique proposes to schedule part of demand traffics
indirectly. For example, port a has 5 demand to port b and 20
demand to port c; port c has 20 demand to port b. Solstice will require
3 configurations for these demands. However, we could let port a send
all 25 demand to port c and let port c send 25 demand to port b. In this
way we reduce one configuration time, which is nontrivial for optical
switches. 

\section{Evaluation Plan and Goals}
\label{sec:goal}
First, we want to evaluate our indirection technique by integer
linear programming (ILP) formulations. In the paper, Liu \textit{et al.\ }~\cite{Liu:2015}
provided a ILP formulation of the scheduling problem without indirection
considerations. We plan to build a formulation based on Liu's formulation
together with indirection consideration. Then we plan to use Gurobi~\cite{gurobi}
ILP solver to solve scheduling problems with the two formulations in order
to find the optimal improvement by indirection. This will be our 75\%
goal.

Then we plan to evaluate our indirection techinque by simulations. We received the
simulator of the Solstice algorithm from the authors. We plan to introduce
our techinque into the simulator and compare it with the original algorithm.
The simulation results will be comparable in terms of number of configurations,
and the total time to serve all demands. This will be our 100\% goal.
Furthermore, we may deploy our heuristic into a real optical switch and evaluate it with real world
workload. This will be our 125\% goal.

